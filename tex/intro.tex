\chapter{绪论}
\section{引言}
随着经济的发展和人类对海洋资源需求的增加,海洋资源开发、海底石油勘探、环境监测、灾难预警等都离不开水声通信技术,且对于通信质量和通信速率的要求也越来越高,与此同时,水声通信网络的需求也越来越大,越来越多的受到重视[1]。相比于无线电信道,水声信道具有快速时变、带宽有限、长多途时延、频率选择性衰落等特点,这些特性严重限制了水声通信和水声通信网络的性能。水声通信网络的设计目标是实现高吞吐量、低能量消耗,在满足固定发射功率或者固定误码率等性能的条件下最大限度的利用信道。因此,为了满足水声通信网络对高吞吐量、低能量消耗等要求,水声自适应链路的研究显得尤为重要。

链路自适应技术是指随着无线信道状态的变化而自动的调节发射机的相关参数以克服由于信道的时变而引起的变化,从而获得最佳通信效果的技术。链路自适应技术可以分为功率自适应控制、自适应调制编码技术和混合自动重传请求技术。通过功率自适应控制可以实现整个网络能量消耗最小的目标;自适应调制编码技术在保证能量消耗一定的情况下实现网络吞吐量的最大化;混合自动重传请求技术结合了前向纠错和自动请求重传技术,能够自动的适应信道的变化,通过对无法正确解码的信息的重新传输来自动适应信道的变化,提高系统的传输性能[2-5]。 链路自适应编码调制可以根据信道情况自适应的调整发射信号参数或者说调整通信速率,使其满足预设的误码率要求。目前对于自适应调制主要调整的参数包括:调制方式、编码速率、发射功率等,还有一些文献将信道功率谱反馈给发射机,发射机根据信道情况自适应调整每个子载波的发射功率,降低信道频率选择性衰落对信号的影响[2-5]。OFDM 有着良好的抵抗多径时延扩展的能力,而且频谱的利用更加紧凑,增加了系统的吞吐能力,很好地适应了当今多媒体通信的需要。因而这种在无线电领域已经相对成熟的多载波调制技术 OFDM 因此被广泛地引入水声通信领域,从而很大程度上弥补了带宽以及信道条件对于通信质量的影响[6-7]。

由于水下信道具有频率选择性和时间选择性,水声信道的信道容量是一个具有时变性的随机变量,要最大限度地利用信道容量,必须使信息的发送速率也是一个随信道容量变化的量,也就是使编码和调制方式具有自适应特性。在水下通信中,资源带宽和功率是非常有限的,为了更加有效地利用无线资源,提高信息的传输速率,需要实时地对信道状况进行监测并以此为依据调整发送方案。比如在恶劣的信道状态下,采用阶数较低的调制方式和低码率的信道编码来达到一定的性能指标,当信道质量有了提高之后,再采用高阶调制和较高的码率。如果信道质量非常良好,甚至可以考虑不采用信道编码[8]。这样,有利的信道条件便得到了适时的利用,从而使信道容量得到更加充分的利用,最终提高频谱效率。 链路自适应技术在陆上无线通信网络中研究比较广泛且较为成熟(如文献[9]、[10]和[11]),已经应用在 802.11 协议中,而在水声通信领域,由于水声信道环境较之无线信道更为复杂多变,导致水声自适应链路的研究困难重重。该方面的研究仍处于起步阶段,且国内外关于该方面的文献也比较少,最早的关于水声自适应链路的文献是2008 年,S. Mani, T.M. Duman  和P. Hursky等人在 Acoustics会议上提出的自适应编码调制技术[12],该文献采用单载波 PSK 调制,通过调整调制阶数和编码速率实现自适应通信,其中编码采用 Turbo编码,同时讨论了 SISO和 MIMO 系统,并提出了两种信道评估准则:achievable  information rate with Guassian inputs 和 post-dqualization SNR,该算法采用了 AUVfest 2007 的实验数据进行了验证;M. Stojanovic 等人给出一种水声自适应通信方案,利用宽带测量信号估计信号的时延扩展长度以及具体的信噪比情况,自适应地改变调制方式及保护间隔的长度从而达到适配效果[13]。G. Ahmed  和 G. Arslan 通过对水下通信中声传播损失的分析,提出一种自适应通信的框架,适时改变子载波的功率、调制方式以及通信带宽来进行自适应调配[14]。C.F. Lin  等人根据通信信道状况,给出了一种自适应功率分配的方案,根据实际需要,确定相应的最高误码率准则,实现了不同误码率准则的水下自适应通信[15]。Shengli Zhou 等人在研究系统反馈延时和估计误差的基础上,提出了一套以目标误比特率为标准的自适应水下通信方案[16]。Lei Wan  等人提出了有效信噪比的概念,以此来作为评价信道质量,通过自适应编码调制使得数据传输速率最大[17]。Beatrice  Tomasi,Laura  Toni 和Paolo Casari 等人利用海试实验数据对可变速率自适应调制的水下系统的性能进行了仿真分析,文中通过相关 Nakagami-m 衰减分布来建立追踪瞬时信噪比信息的模型,通过仿真验证在充足的参数估计时间里,相关的信道信息可以被所建立的模型准确的表示出来。在频谱利用率,无误码和系统吞吐量等方面,该模型具有重现自适应理论分析仿真结果的能力[18];Andreja Radosevic等人在固定误码率门限实现系统最大的吞吐量的基础上,提出了两种自适应调制方案:一是当发射功率子载波之间是非均匀分布的,只自适应地改变子载波的调制方式;二是发射功率和调制方式都改变。在可以提前预测估计信道的前提下,反馈信息的内容分为两种:一是反馈估计的信道信息;二是在接收端根据估计的信道信息制定好子载波的调制方式和功率分配,将此信息反馈回到发射端。通过仿真和实验证明了在时变信道的水声信道中可以提前一个信道周期预测信道,自适应方案可以提高频谱利用率和系统的吞吐量,为水下高速可靠的通信提供了可能[19];国内研究中,沈晓红、黄建国等人研究了舰船之间的多制式正交多载波自适应通信技术,在湖试中达到了水声通信中常规的 40km?kb/s  系统性能指标[20];上海交通大学的周卫星进行了水声通信中基于 HARQ 的链路自适应技术的研究。文章针对时不变水声信道模型和时变的 Nakagami-m 水声信道衰落模型,分析了使用三种不同类型HARQ 技术时水声信道期望吞吐率。仿真结果表明,对于带宽有限、时延较大的水声信道,经过自适应调制编码技术处理后的链路吞吐率性能优于固定调制编码方案的吞吐率性能的计算方法;并且对点对点的链路模型进行扩展,分析研究 HARQ  技术和自适应调制编码技术在水声中继网络中的应用。最后研究表明,源节点和目的节点中间位置的中继节点具有更好的吞吐率性能[21];哈尔滨工程大学的陈伟林研究了 OFDM 自适应编码调制的水声通信技术,并通过硬件 DSP进行了实现。在水池实验中,信噪比为 8dB时,误码率达到 3.23\%,通信速率达到3028bps。

\section{自适应OFDM通信的发展及研究现状}
\section{水声信道的特点}
\section{论文的主要研究内容}