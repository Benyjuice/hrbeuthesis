\begin{tikzpicture}
\draw 
	node[left] at (2,-2) {\xiaowu 发射}
	node[right] at (8,-3) {\xiaowu 接收};
\draw plot[smooth,tension=.9] coordinates {(0.000000,0.292632) (0.500000,0.274862) (1.000000,0.458597) (1.500000,0.142920) (2.000000,0.378600) (2.500000,0.376865) (3.000000,0.190223) (3.500000,0.283911) (4.000000,0.037927) (4.500000,0.026975) (5.000000,0.265399) (5.500000,0.389584) (6.000000,0.467005) (6.500000,0.064953) (7.000000,0.284412) (7.500000,0.234695) (8.000000,0.005951) (8.500000,0.168561) (9.000000,0.081091) (9.500000,0.397142) (10.000000,0.155608)};

\draw plot[smooth,tension=.9] coordinates {(0.000000,-4.735733) (0.500000,-4.917176) (1.000000,-4.699009) (1.500000,-4.868514) (2.000000,-4.672960) (2.500000,-4.655393) (3.000000,-4.625924) (3.500000,-4.774729) (4.000000,-4.958089) (4.500000,-4.885512) (5.000000,-4.543331) (5.500000,-4.923811) (6.000000,-4.587092) (6.500000,-4.730829) (7.000000,-4.501933) (7.500000,-4.960912) (8.000000,-4.778661) (8.500000,-4.946674) (9.000000,-4.519051) (9.500000,-4.997683) (10.000000,-4.612545) };

\draw[->] (2,-2) --node[below] {\xiaowu 海底反射路径} (5,-4.54) --  (8,-3);
\draw[->] (2,-2) -- node[above] {\xiaowu 直达路径} (8,-3);
\draw[->] (2,-2) -- (4.5,0.027) --node[above] {\xiaowu 海面反射路径} (8,-3);
\end{tikzpicture}