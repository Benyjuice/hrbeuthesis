\def \nod{~~~~}
\def \zihao{\wuhao}
	\tikzstyle{neuron}=[circle,fill=black!25,minimum size=0pt,inner sep=0pt]
	\tikzstyle{num}=[circle,fill=red!25,minimum size=0pt,inner sep=0pt]
	\tikzstyle{vecArrow} = [thick, decoration={markings,mark=at position
		1 with {\arrow[semithick]{open triangle 60}}},
	double distance=1.4pt, shorten >= 5.5pt,
	preaction = {decorate},
	postaction = {draw,line width=1.4pt, white,shorten >= 4.5pt}]
	\tikzstyle{adct} = [rectangle,fill=gray!25,minimum width=5cm, minimum height=1cm,text centered, draw=black]
	\tikzstyle{block} = [rectangle,fill=gray!75!red,minimum width=2cm, minimum height=1cm,text centered, draw=black]
\begin{tikzpicture}[>=stealth',node distance=3cm]
	\draw 
		node[adct] (adc) {ADC数据采集驱动}
		node[adct,xshift=6cm] (uart) {UART数据传输驱动}
		node[block,above of=adc,xshift=-1.5cm] (sync) {同步}
		node[block,right of=sync,xshift=1cm] (cache) {缓存}
		node[block,above of=uart] (pro) {解扩}
	;
	\node[inner sep=0,minimum size=0](hid) at (-1.5cm,0.5cm){};
	\node[inner sep=0,minimum size=0,right of=hid,xshift=-2.98cm,yshift=1cm](k) {};
	\draw [vecArrow] (hid) -- (sync);
	\draw [vecArrow] (k) -| (cache);
	\draw [vecArrow] (cache) -- (pro);
	\draw [vecArrow] (pro) -- (uart);
	\draw [vecArrow,dashed] (sync) -- (cache);
	
	\draw[dashed,very thick] (-4cm,1.0cm) --++(14cm,0);
	\node at (9.5cm,1.4cm) {应用层};
	\node at (9.5cm,0.6cm) {驱动层};
	
\end{tikzpicture}