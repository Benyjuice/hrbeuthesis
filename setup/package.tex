% -*-coding: utf-8 -*-
% File: setup/package.tex
%
% 修改自: PlutoThesis_UTF8_1.9.4.20100419.zip
%         http://code.google.com/p/plutothesis/
%
%
%%% 可能用到的宏包以及参数定义



%% 图形支持宏包,为了使用 pdftex,需要作相应判断
\usepackage{etex}  % 增加计数器总数(原来是 256,宏包多,可能不够用);编译需基于 eTeX,目前 pdflatex 等均可
\usepackage{ifpdf}

\ifx\atempxetex\usewhat  % main.tex: \def\atempxetex{xelatex}
	\usepackage[dvipdfm]{graphicx}
\else
	\ifpdf
		\usepackage[pdftex]{graphicx}
	\else
		\usepackage[dvips]{graphicx}
	\fi
\fi

%% 版面控制宏包,定义规定的版面尺寸
% 纸张大小: \documentclass 的选项中已设置 a4paper,故不为 geometry 设置 paperwidth 和 paperheight
% 若设置 paperwidth=18.4cm, paperheight= 26cm,使用 16 开纸打印,可节能减碳 :)
%
% 页边距和页眉页脚: XXX 哈工程本科论文规定: 
%     打印在 A4 (大 16 开, 210mm*297mm) 纸上,上下左右页边距分别为:45mm、40mm、25mm、45mm
%     上下右三面裁切后,为 16 开纸,184mm*260mm;
%     页眉 38mm,页脚 33mm (指页眉页脚距离纸张边缘的距离);
%       故: top=45mm, bottom=40m, left=25mm, right=45mm
%           totalbody={140mm, 226mm} (width=140, height=226mm)
%           headheight=7mm, headsep=top-headheight-38mm=0mm, footskip=7mm
%           body={140mm*212mm} (textwidth=140, textheight=212mm)
%       因此可设置: top=45true mm, bottom=40true mm, left=25true mm, right=45true mm
%             或者: body={140true mm, 212true mm}
%           数字后跟 true,如 15true cm 意思是保证尺寸不会随着一些参数变化而变化
%
% 其他设置:
% bindingoffset 用于设置页边装订距离;此处设置为0,使用论文规范中定义的左边距 25 mm 来装订
% 页码在版芯下边线之下隔行居中放置;TODO
%
% XXX: 经过对现行 Word 模板进行实际测量,这里将 top 下压为 48mm,相应的 headsep 增大为 3mm
% 
% 老旧参数,headsep 为零使得页眉线和正文距离太近
%\usepackage[top=45true mm, bottom=40true mm, left=25true mm, right=45true mm,
%            headheight=7mm, headsep=0mm, footskip=7mm,
%            bindingoffset=0 pt
%            ]{geometry}
\usepackage[top=48true mm, bottom=40true mm, left=25true mm, right=45true mm,
            headheight=7mm, headsep=3mm, footskip=7mm,
            bindingoffset=0 pt
            ]{geometry}
\usepackage{layouts}                    % 打印当前页面格式的宏包
\usepackage[sf]{titlesec}               % 控制标题的宏包,sf 表示无衬线体,中文即黑体
\usepackage{titletoc}                   % 控制目录的宏包,说明文档与 titlesec 在一起
\usepackage[perpage,symbol]{footmisc}   % 脚注控制
\usepackage{fancyhdr}                   % 页眉和页脚的相关定义
\usepackage{fancyref}                   % 设置不同的引用文字
\usepackage{array}                      % 增强表格的功能,可能与xeCJK宏包冲突,需放在xeCJK之前

\ifx\atempxetex\usewhat       % 如果使用 xelatex
%\usepackage[slantfont,boldfont,CJKaddspaces]{xeCJK}
%\CJKlanguage{zh-cn}
\usepackage{ctex}
\else
%\usepackage{CJKutf8} %CJKpunct}
\usepackage[UTF8]{ctex}
\usepackage{times}            % 使用 Times 字体的宏包
\fi                          
                             
\usepackage{type1cm}          % tex1cm 宏包,控制字体的大小
\usepackage{indentfirst}      % 首行缩进宏包
\usepackage{color}            % 支持彩色
                             
\usepackage{amsmath}          % AMSLaTeX 宏包,用来排出更加漂亮的公式
\usepackage{relsize}          % 调整公式字体大小: \mathsmaller \mathlarger
\usepackage{amssymb}         
\usepackage{textcomp}         % 千分号等特殊符号
\usepackage{mathrsfs}         % 不同于 \mathcal or \mathfrak 之类的英文花体字体
\usepackage{bm}               % 处理数学公式中的黑斜体的宏包

% 定理类环境宏包,其中 amsmath 选项用来兼容 AMS LaTeX 的宏包
\usepackage[amsmath,thmmarks,hyperref]{ntheorem}

\usepackage{epsfig}           % eps 图像
\usepackage[below]{placeins}  % 允许上一个 section 的浮动图形出现在下一个 section 的开始部分,还提供 \FloatBarrier 命令,使所有未处理的浮动图形立即被处理
%\usepackage{psfrag}          % 替换 eps 图形中的文字
%\usepackage{floatflt}        % 图文混排用宏包,texlive2008 默认不带此宏包,可以自己安装,若是不需要这个功能可以不管它。
\usepackage{rotating}         % 图形和表格的控制
%\usepackage{endfloat}        % 可将浮动对象放置到文件的最后
\usepackage{setspace}         % 定制表格和图形的多行标题行距
\usepackage{flafter}          % 使得所有浮动体不能被放置在其浮动环境之前,以免浮动体在引述它的文本之前出现.
\usepackage{multirow}         % 使用 Multirow 宏包,使得表格可以合并多个 row 格
\usepackage{booktabs}         % 表格,横的粗线;\specialrule{1pt}{0pt}{0pt}
\usepackage{longtable}        % 支持跨页的表格

\usepackage[hang,center]{subfigure}  %支持子图;centerlast 设置最后一行是否居中
\usepackage[subfigure]{ccaption}  % 浮动图形和表格标题样式,已经替代 caption2

%\usepackage{cite}            % 支持引用的宏包
\usepackage[sort&compress,numbers]{natbib}  % 支持引用缩写的宏包
\usepackage{hypernat}

\usepackage{enumitem}         % 使用 enumitem 宏包,改变列表项的格式
\usepackage{calc}             % 长度可以用 + - * / 进行计算

% 生成有书签的 pdf 及其开关, 该宏包应放在所有宏包的最后, 宏包之间有冲突
\def\atemp{dvipspdf}
\ifx\atemp\usewhat
% XXX: 打印的时候可以将下面的 colorlinks 改成 false,使字体都是黑色
% 设置 pdfborder 用于去掉链接的边框
\usepackage[dvips,unicode,
           bookmarksnumbered=true,
           bookmarksopen=true,
           colorlinks=false,
           pdfborder={0 0 1},
           citecolor=blue,
           linkcolor=red,
           anchorcolor=green,
           urlcolor=blue,            
           breaklinks=true
           ]{hyperref}
\usepackage{breakurl}         % 消除 dvips 的时候网页链接断行失效的问题
\fi

\def\atemp{dvipdfmx}
\ifx\atemp\usewhat
% dvi-->pdf 生成书签
\usepackage[dvipdfm,unicode,
            bookmarksnumbered=true,
            bookmarksopen=true,
            colorlinks=false,
            pdfborder={0 0 1},
            citecolor=blue,
            linkcolor=red,
            anchorcolor=green,
            urlcolor=blue,
            breaklinks=true
            ]{hyperref}
\fi

\def\atemp{pdflatex}
\ifx\atemp\usewhat

\usepackage{cmap}             % pdflatex 编译时,可以生成可复制、粘贴的中文 PDF 文档
\usepackage[pdftex,unicode,
            %CJKbookmarks=true,
            bookmarksnumbered=true,
            bookmarksopen=true,
            colorlinks=false,
            pdfborder={0 0 1},
            citecolor=blue,
            linkcolor=red,
            anchorcolor=green,
            urlcolor=blue,           
            breaklinks=true
            ]{hyperref}
\fi

\def\atemp{yap}
\ifx\atemp\usewhat
% 考虑到 yap 可以正反向搜索,dvipdf 使 yap 中的链接有效
\usepackage[dvipdfmx,unicode,
            %CJKbookmarks=true,
            bookmarksnumbered=true,
            bookmarksopen=true,
            colorlinks=false,
            pdfborder={0 0 1},
            citecolor=blue,
            linkcolor=red,
            anchorcolor=green,
            urlcolor=blue,       
            breaklinks=true
            ]{hyperref}
\fi

\ifx\atempxetex\usewhat
% 下面的 naturalnames 用于与 algorithm2e 宏包协调
\usepackage[xetex,
            bookmarksnumbered=true,
            bookmarksopen=true,
            colorlinks=false,
            pdfborder={0 0 1},
            citecolor=blue,
            linkcolor=red,
            anchorcolor=green,
            urlcolor=blue,
            breaklinks=true,
            naturalnames
            ]{hyperref}  
\fi

\usepackage{arydshln}        % 分块矩阵画虚线,挺好用

% 算法的宏包,注意宏包兼容性,先后顺序为float、hyperref、algorithm(2e),否则无法生成算法列表
\usepackage[boxed, linesnumbered, algochapter]{algorithm2e}
