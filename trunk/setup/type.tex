% -*-coding: utf-8 -*-
% File: setup/type.tex
%
% 修改自: PlutoThesis_UTF8_1.9.4.20100419.zip
%         http://code.google.com/p/plutothesis/
%
% 修改记录: Yuliang <jyl198803@gmail.com> 2010-05-23
%     设计哈工程本科毕业论文规范
%
% 注: 标注 TODO 的是暂未测试的内容,或者计划将来完成的内容;
%
%

% 本硕博类型的一些定义

%% 导言区使用中文
\makeatletter % 将 @ 符号当作字母,以便使用 LaTeX 内部命令
\@tempcnta=128 % 定义变量 \@tempcnta 为 128
\loop % 开始一个循环
  \catcode\@tempcnta=13
  \ifnum\@tempcnta<255
    \advance \@tempcnta \@ne
  \repeat
\makeatother % 恢复 @ 符号的原义

%% 判断论文类型
\newif\ifxueweidoctor % 声明三个新的逻辑型变量: \xueweidoctor \xueweimaster \xueweibachelor
\newif\ifxueweimaster
\newif\ifxueweibachelor

% 根据 \xuewei 的定义为 \xueweidoctor \xueweimaster \xueweibachelor 赋值
\def\temp{Doctor}
\ifx\temp\xuewei % 若 \temp 和 \xuewei 相匹配
  \xueweidoctortrue  \xueweimasterfalse \xueweibachelorfalse
\fi
\def\temp{Master}
\ifx\temp\xuewei
  \xueweidoctorfalse  \xueweimastertrue \xueweibachelorfalse
\fi
\def\temp{Bachelor}
\ifx\temp\xuewei
  \xueweidoctorfalse  \xueweimasterfalse \xueweibachelortrue
\fi

% 定义学位相关的几个命令
\ifxueweidoctor
  \newcommand{\cxuewei}{博士}
  \newcommand{\exuewei}{Doctor}
  \newcommand{\exueweier}{Doctoral}
  \newcommand{\xueweishort}{博}
\fi

\ifxueweimaster
  \newcommand{\cxuewei}{硕士}
  \newcommand{\exuewei}{Master}
  \newcommand{\exueweier}{Master}
  \newcommand{\xueweishort}{硕}
\fi

\ifxueweibachelor
  \newcommand{\cxuewei}{本科生}
  \newcommand{\exuewei}{Bachelor}
  \newcommand{\exueweier}{Bachelor}
  \newcommand{\xueweishort}{本}
\fi

% 决定单双面打印
\ifxueweidoctor % TODO: 哈工程博士?面打印
  \def\oneortwoside{oneside}
\fi

\ifxueweidoctor % TODO: 哈工程硕士?面打印
  \def\oneortwoside{oneside}
\fi

\ifxueweibachelor % 哈工程本科论文单面打印
  \def\oneortwoside{oneside}
\fi

% 修改了上面的学位判断逻辑,\oneortwoside 没有定义的情况应该不会发生了
\ifx\oneortwoside\undefined
  \def\oneortwoside{oneside}
\fi

% 定义逻辑变量 \oneortwoside,根据前面的 \oneortwoside 变量内容决定 true or false
% twoside 为 true
\newif\ifoneortwoside

\def\temp{twoside}
\ifx\temp\oneortwoside
  \oneortwosidetrue
\else
  \oneortwosidefalse
\fi
