% -*-coding: utf-8 -*-
% File: preface/cover.tex
%
% 修改自: PlutoThesis_UTF8_1.9.4.20100419.zip
%         http://code.google.com/p/plutothesis/
%
% 修改记录: Yuliang <jyl198803@gmail.com> 2010-05-25
%     设计哈工程本科毕业论文规范
%
% 注: 标注 TODO 的是未测试的内容,或者计划将来完成的内容;
%     目前专门针对 Linux 进行修改,由于论文规范的字体要求,使用 Windows 的盗版字体;
%

\newcommand{\chinesethesistitle}{哈尔滨工程大学本硕博学位论文~\LaTeX~模板~(\version~版)}  % 授权书用,无需断行
\newcommand{\englishthesistitle}{\uppercase{\LaTeX~Dissertation Template of \\Harbin Engineering University~(Version \version)}}  % \uppercase作用:将英文标题字母全部大写
\newcommand{\chinesethesistime}{2010~年~6~月}  % 封面底部的日期中文形式
\newcommand{\englishthesistime}{June, 2010}   % 封面底部的日期英文形式

\ctitle{哈尔滨工程大学本硕博学位论文\\ \LaTeX~模板~(\version~版)}  % 封面用论文标题,自己可手动断行
\cdegree{\cxueke\cxuewei}
\csubject{计算机科学与技术}
\caffil{计算机科学与技术学院} %(在校生填所在系名称,同等学力人员填工作单位)
\cauthor{某某某}  % 或{某~~~~某}
\csupervisor{某某某~~~~教授}  % 导师名字
%\cassosupervisor{某~~~~某~~~~教授}  % (如无副导师可以不列此项)
%\ccosupervisor{某某某~~~~教授~}  % (如无联合培养导师则不列此项)
\cdate{\chinesethesistime}

\etitle{\englishthesistitle}
\edegree{\exuewei \ of \exueke}
\esubject{Microelectronics \hfill and \hfill Solid-State\newline Electronics}  % 英文二级学科名
\eaffil{Dept.\hfill of\hfill Microelectronics\hfill Science\newline and Technology}  % 英文单位,换行用 \newline,不要用\\
\eauthor{Alice}  % 作者姓名
\esupervisor{Professor Bob}  % 导师姓名
%\ecosupervisor{Professor X}
%\eassosupervisor{Professor Y}
\edate{\englishthesistime}

\natclassifiedindex{TP309}  % 国内图书分类号
\internatclassifiedindex{681.324}  % 国际图书分类号
\statesecrets{公开} %秘密

%\iffalse
%\BiAppendixChapter{摘~~~~要}{}  %使用winedt编辑时文档结构图(toc)中为了显示摘要,故增加此句;
%\fi
\cabstract{
这是根据哈尔滨工程大学本科论文规范制作的\LaTeX{}本硕博学位论文模板。目前暂时未制作硕博部分,
未来会把这些部分都添加上去。

本模板是网友 Yuliang 等(2010)基于哈尔滨工业大学硕博士学位论文模板,按照哈尔滨工程大学相关论
文规范制作的\LaTeX{}论文模板,论文的启动和制作过程得到了工大论文模板维护者 luckfox 的大力支
持,在此表示感谢。

当然这个模板文件仅仅是一个开始,
希望有``牛人''能够综合这些设置形成真正的文档类形式(cls)的模板文件,造福以后的兄弟姐妹们。
不过补充一下, 在目前需要多人参与维护的情况下,book类的文档也具有一些自己的优势,
大家都很容易看懂代码,上手修改。二者各有特色吧。
总体上来说,当前这个模板还是很值得推荐使用的。

本模板的目的旨在推广\LaTeX{}这一优秀的排版软件在哈工程的应用,为广大同学提供一个方便、美观的
论文模板,减少论文撰写方面的麻烦。
}

\ckeywords{\LaTeX{};论文模板;哈工程;(3 $\sim$ 6 个!)}

\eabstract{
This is a \LaTeX{} dissertation template of Harbin Engineering University, which is built according to the required format.
}

\ekeywords{\LaTeX{}; dissertation template; (attention:3 $\sim$ 6 key words,lower-case!)}

\NotationList{\begin{tabular}{ll} %主要符号表
A & a matrix\\
B &  登高\\
\end{tabular}}

\makecover
\clearpage
