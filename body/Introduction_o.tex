% ^^^^^^ This is for TeXworks users ^^^^^
% !TEX root = ../main.tex 
% !TEX program = xelatex
% !TEX encoding = utf-8
% vvvvvv This is for TeXworks users vvvvv

% -*-coding: utf-8 -*-


\defaultfont

\BiChapter{ 绪论~$H_{\infty}$ }{ Introduction ~$H_{\infty}$ }
\label{Introduction}

\BiSection{课题背景及意义~$H_{\infty}$ }{ The Background and Significance $H_{\infty}$ }
\label{Introduction:background}
\LaTeX~由于具有排版美观、对公式和图表的处理能力强大以及跨平台通用性强等优势,
使得它在科技排版中的应用越来越广泛。

\BiSection{入门知识}{Mastering \LaTeX{}}
\label{sec:learningknowledge}
考虑到不少同学没有接触过~\LaTeX{},为了少走弯路,快速上手,尽快学会~\LaTeX{}的基本使用方法,
从而把更多的时间投入到论文的写作过程中,更专注于论文的内容,特增加一节介绍~\LaTeX 的基本知识,
推荐一些文档资料,及常用的编辑技巧等内容。

\BiSubsection{什么是\LaTeX{}}{What is \LaTeX}
\label{sec:whatislatex}
        \TeX/\LaTeX 是一套功能强大、排版完美的开放源程序的免费办公排版软件。

        对多种操作系统,包括~Microsoft Windows、\CJKglue Unix类~(如:Solaris、\CJKglue Linux 等)、\CJKglue
以及~Mac OS X 均供有相应的运行版本,其名字也不尽相同,其发展过程类似于基于~linux 内
核的众多~linux 操作系统的发展过程。

        在~Windows 下最常用的是~\href{http://www.miktex.org}{MikTeX} 及其衍生出来的套装。Linux 下现在最常用并且持续更新的是
        TeXlive(跨平台,某些版本也可用在windows下),另一个编译系统~teTeX 最近停止了维护。

        在MikTeX基础上,\href{http://www.ctex.org}{CTeX} 的~Aloft 站长加入了中文输入输出支持,配置了~CTeX 中文套
装,安装即用,免去了用户的配置之苦,推荐中文用户入门使用。

        与所见即所得~(WYSIWYG,What You See Is What You Get) 的~Microsoft Office
软件相比,它的特点是:
   \begin{itemize}
      \item 所想即所得~(WYSTWYG,What You Think Is What You Get),让你更专注
于论文的思路贯通而不是繁杂的格式要求,更适合排版科技论文;
      \item 控制格式方便,键盘输入快捷,数学公式输入排版方便,输出精美;
      \item 纯文本文件避免了类似~MSWord 的各种格式易变、文档损坏、公式无法编辑等不稳定现象,
      也更有利于版本控制;
      \item 输出的~PDF 文件是国际文档标准,哈工大要求硕博士毕业论文提交的也是~PDF 格式;
      \item 国际期刊及会议一般都提供~\LaTeX 论文模板,使论文投稿排版更容易;
      \item 目前国内外不少高校也都具有~\LaTeX{}学位论文写作模板,使写作学位论文的排版不再是痛苦,
      而是一种享受。哈工大比较成熟的即为此模板;
      \item 制作幻灯片的~LaTeX 宏包~beamer,排版公式和输入文字一样方便,没有~PowerPoint
的那种繁琐公式和图片位置调整,众多的默认模版供选择,一个简单命令就可切换,让幻灯
片制作更轻松、专业、漂亮;
      \item  众多的文档类和宏包支持,给你的感觉是``没有你做不到的,只有你想不到的'';
      \item  对许多忠实的~TeXer 而言,\TeX/\LaTeX 已经不仅仅是一种排版软件,更成为一种信仰,
      因为它的诞生及其发展本身就是一段趋向完美的传奇。
    \end{itemize}

    很多人都对~\LaTeX{} 做过介绍,你可以从紫丁香的TeX版编号为~ 1 的帖子往后翻上几页看看。校内的RunFTP上的目录
    \url{ftp://202.118.224.241/software/Science/TeX&LaTeX/\%D7\%CA\%C1\%CF/LaTeX/} 上也有几个幻灯片对它进行了介绍。
    本文作者反复修改多次,最后决定推荐网站~\href{http://zzg34b.w3.c361.com/index.htm}{LaTeX编辑部} 上的几篇文章,快速全面了解它。
   \begin{itemize}
     \item  \href{http://learn.tsinghua.edu.cn:8080/2001315450/tex_frame.html}{TeX简介}:我们使用的~\LaTeX 系统的基础;
     \item \href{http://zzg34b.w3.c361.com/homepage/TeXvirtue.htm}{TeX的优缺点}: 它趋向完美,但是还不是十全十美;
     \item \href{http://zzg34b.w3.c361.com/homepage/LaTeXbring.htm}{LaTeX的产生}:我们日常接触最多的是它;
     \item \href{http://zzg34b.w3.c361.com/homepage/compareWord.htm}{LaTeX应用情况}:国内外的情况,现在国内是使用者越来越多。
     \item \href{http://zzg34b.w3.c361.com/homepage/compareWord.htm}{与Word相比较}:对于我们用的软件~word和~LaTeX,二者各有优点和缺点,不要陷入很偏执的观点中,这是一篇比较客观的文章。
     \item \href{http://zzg34b.w3.c361.com/homepage/KnuthResume.htm}{Knuth 教授简历}:底层内核~ \TeX 的作者~ Donald Knuth (高德纳)的介绍,富有传奇色彩;
     \item \href{http://zzg34b.w3.c361.com/homepage/LamportResume.htm}{Lamport 博士简历}: \LaTeX 的作者~Lamport,由于他的努力,才让~\LaTeX 使用简单很多,风靡科技界;
   \end{itemize}

\BiSubsection{推荐软件及其下载}{LaTeX Software}
\label{sec:latexsoftware}

对于新手,Windows 下只推荐免费的~CTeX 套装,因为丝毫不需要自己配置,安装后就可以使用。
哈工大校园网用户可以从~ \url{ftp://202.118.224.241/software/Science/TeX&LaTeX/\%C8\%ED\%BC\%FE/TeXSystem/CTeX} 下载~
CTeX-2.4.5-8-Full.exe 和~ CTeX-Fonts-2.4.4.exe,先安装套装系统,然后安装字体。当然,也可以下载同目录下的最新版ctex 2.4.6的basic或full
版本,别忘了安装fonts-2.4.6文件。非校园网用户可以从~ \href{http://www.ctex.org}{CTeX的官方网站} 下载这两个文件。需要说明的是,CTeX 2.4 基于 miktex 2.4,支持PlutoThesis的dvipspdf,dvipdfmx和pdflatex的编译方式,不支持xelatex编译方式,还请注意。若想使用xelatex进行编译,需 miktex 2.7版,在此基础上加上dvipspdf,dvipdfmx和pdflatex中文处理的是 instanton@ctex 的MiCTeX 版本,可以在ctex论坛上找到发布和下载信息,241 ftp上可能也有不保证最新的版本下载。


\BiSubsection{推荐的入门资料}{LaTeX Documents}

如果你是第一次接触~ LaTeX,那么安装~ CTeX 之后,不要直接打开编辑软件~WinEdt 进行操作,
因为现在你对这个软件了解还较少,会无所适从。请从~ Windows
系统的开始~ $\rightarrow$ 程序~ $\rightarrow$ 中文 TeX 套装~ $\rightarrow$ help $\rightarrow$
看到文档了吧,我们建议的顺序是:首先打开~ CTeX FAQ,从头读到''新手入门''这一节,然后
从同一目录下打开~ LShort-cn 文件,从头到尾认真浏览一遍,不用尝试记住所有的命令,
只要了解~ LaTeX 的特点,对每一部份有一个感性认识就可以。以后你还可以回头来翻看相关命令。
然后把 ~CTeX FAQ 浏览完毕,在这个过程中,可以尝试去练习排版一些文档,查看编译效果(具体~ WinEdt 编译
方法请看后面的节 \ref{sec:winedttricks})。

你会留意到还有两个文档~PDF: latex2e 插图指南和~mathematics (LaTeX 宝典 The LaTeX Companion 的~chapter8),
分别是讲插图知识和公式输入方法的,都很详细,建议也浏览一下。对于公式输入,还有一个特别值得推荐的是
\href{http://www.tug.org/tex-archive/info/math/voss/mathmode/}{mathmode 2.0},校园网用户还可以从
\href{ftp://202.118.224.241/software/Science/TeX&LaTeX/}{校内~FTP TeX 资料目录}下载。

\BiSubsection{WinEdt的编译及其他技巧}{Winedt Tricks}
\label{sec:winedttricks}
~
有一文档~ WinEdt\_LaTeX\_guide.doc 简单介绍了~ WinEdt 的简单文档的编译方法,可以点击
\url{http://bbs.hit.edu.cn/bbscon.php?bid=296&id=1887&ap=719} 得到。

下面详细介绍编译按钮的含义,新手一般对这个特别好奇,
请注意这里的讲解顺序不是~ WinEdt 的默认排列顺序。
\begin{hitlist}
  \item TeX: 用来编译使用~ TeX 命令写的文档,是底层的编译系统;
  \item LaTeX: 用来编译使用~ LaTeX 命令写的文档,是目前我们使用最多的~ LaTeX2e 文档编译系统,生成~ dvi 文件;
  \item cct\& LaTeX: cct 是国内的张林波研究员开发的一个使用~ LaTeX 来处理中文文档的接口系统,
  首先把~cct 的文档由~.ctx 转换成~.tex 格式,然后调用标准的~ LaTeX 命令来生成dvi文件;
  \item PDFLaTeX: 这是类似于~LaTeX的另外一种编译系统,直接生成~pdf 文件,支持更多的~ pdf 文件特效,现在应用越来越广泛,例如做的幻灯片;
  \item BibTeX: 这个是用来处理参考文献的命令,通过它生成一个包含参考文献条目的列表~ bbl 文件供排版使用。
  \item Make Index: 这个用来生成文档的索引。
  \item TeXify: 这是几个编译命令的合集,它自动运行~ LaTeX(或pdflatex),MakeIndex 和~ BibTeX 尽可能需要的次数来生成一个
  具有排序的文献列表和交叉引用的~ dvi(pdf)文件,简化了~ dvi(pdf)文件的生成过程。
  \item CTeXify: 这个是~ CTeX 套装添加了中文支持的~TeXify 命令,可以生成中文的~ dvi(pdf) 文档。
\end{hitlist}

具体使用哪个编译按钮,和你的文档类型及包含的内容有关系。因为不同的编译命令,对于文档中的元素要求不一样,
例如,如果你引用的是~eps 图形,应该用~latex来编译,如果插入的是~pdf 图形,应该用~pdflatex 来编译。
这个解释你在前面的文档里应该也已经看到了。

\BiSubsubsection{显示文档结构图}{File Structure Display}

WinEdt中的~gather可以收集章节标题,形成~TOC 列表,功能类似于~word 中的文档结构图,在
写大文档的时候这个功能非常有用,但是在我们的~Pluto 模板中,自定义了一些章节标题,
这些自定义标题缺省的~gather 是不识别的。TeX@lilac 提供了一种方法,在~WinEdt.gdi中定义了相关命令,
奏效。希望想使用这个功能的网友可以自己动手修改,tools 文件夹下有他修改过的~WinEdt.gdi,
网友们也可以直接使用,放到~winedt 目录下替换同名文件即可。

winEdt 的~ tree interfacezho 中也有~ TOC 这一项,这个可以通过修改 ~Winedt目录下的~WinEdtEx.ini实现。用~tools文件夹下
的~WinEdtEx.ini 替换该同名文件就可以。

另外这些自定义的命令在编辑状态下不能像缺省命令一样高亮,如果能高亮就好了,TeX 同样找到了自己定义的方法,
在~winedt 菜单~ option/highlighting/switches 修改,tools 文件下~ Switches.dat 是他已
定义好的,可以在上述菜单位置使用对话窗顶部的~``Load from'' 按钮加载。

\BiSubsection{生成图的常见方式}{Figure Generating}

在~latex 文档编写过程中,常用的图形格式是~ eps 和~ pdf 。

pdf 文件生成,可以用很多软件生成,例如~adobe acrobat 、pdffactory、pdf xchange 等。
这里推荐~ acrobat (注意不是~ acrobat  reader),因为它安装后生成一个~ pdf 打印机,
任何一个文档都可以通过这个打印机生成~pdf 文件,更主要的功能是对~ pdf 文件阅读编辑。
pdf 文件可以在~acrobat 里进行裁剪~ (documents,crop pages...),取出部分页面。它还支持
直接另存为~ eps 文件的功能。所以,有了~acrobat 软件,几乎所有的图形处理问题都可以解决。
不过这个软件是商业软件。

可用于~latex 的作图软件比较多,所有的作图软件:~visio,~ coraldraw,~ photoshop,~ gnuplot 生成的图
都可以通过上面的方法转换成~eps 或~pdf 文件。另外,还有很多专门为~latex 开发的作图软件,
有通过命令的形式生成图的,例如~metapost,~pstricks,~asymptote,~pgf/tikz 等,也有简单的
具有作图界面的软件,例如~Dia,~winfig,~gclc 等。  总体上来讲,这些软件功能相对比较简单,做出的图也都很漂亮,但是都没有微软的
~visio 软件功能强大,所以对于入门者最好先用~visio 来作图。

对于插图的命令,王磊的中文版插图指南已经很详细了,这里不再介绍,请翻阅该书。

\BiSubsection{新手如何写数学公式}{Math Inputting}

新手一般感觉用~LaTeX 写公式比较麻烦,从而望而却步。其实,你可以从~mathtype(商业软件,
如果你在~ word 里不用它的话,可以用它免费的~TeX 公式软件~TeXaids)转换的方法。
在~mathtype 公式输入界面下,选择菜单栏 ~``preference,translators...,translate to other language,
选择~ TeX--LaTeX2.09 and later (标准的LaTeX命令),或者~ TeX--AMS-LaTeX(需要~amsmaht 宏包支持) ,确定 ''
,然后输入公式,选中,复制,粘贴到你的编辑器中你想放入公式的地方,你会看到这就是你
想要的公式的~ LaTeX  代码,这样就可以正常编译了,有时候你需要调整一下你使用的数学环境才能达到
你的要求。

如果在输入数学公式的时候,你忘记了一个数学符号,可以点击工具栏的求和符号图标,会显示
数学符号工具栏,然后点击你想要的数学符号,就插入到了文档中。还有一个专门的~symbol.pdf 文档,ctex
套装中也已经配备,几乎所有的符号,包括一些稀奇古怪,甚至都没看到过的符号,都可以在这里找到。

前面提到,ctex 套装的~help 文件里已带的~ch8.pdf 和~Mathmode.pdf 文件对公式书写介绍的很详细,各种各样的公式
都可以通过不同的数学环境得到,建议写公式时参阅。

\BiSubsection{快速插入图表}{Inserting Figures and Tables}

对于图表等环境的插入,WinEdt 提供了很体贴的方法。只要选择工具栏的图片和表格按钮,就可以
插入一段完整的图表命令,你需要的只是把其中的星号换成自己的东西就可以了。
WinEdt还提供了一个宏,GUI 方式完成图的插入过程,选项更多,也更方便。
有时候,感觉表格比较复杂,不容易用~LaTeX 写命令,可以尝试一下 ~xl2latex 2.0 这个~ excel
输出表格为~LaTeX 代码的宏。首先在~excel 里生成表格,然后运行一下宏,就生成了表格的~LaTeX 代码。
模板的~edittools 目录下提供了这一文件。

\BiSubsection{更多技巧及别人入门心得}{More Tricks and Others' Thoughts}

更多的入门技巧,可以参考紫丁香TeX版的置底的帖子~\href{http://bbs.hit.edu.cn/bbscon.php?board=TeX&id=2038&ftype=11}{版面部分问题导航(主要是面向新手)}。

\BiSubsection{校内TeX资源}{TeX resources in HIT}
前面已经多次提到RunFTP上的TeX资源,这里再次总结一下校内的这两个FTP。
\begin{hitlist}
  \item ftp://202.118.224.241/software/Science/TeX\&LaTeX/  ,这个为HIT的官方出资,网友维护的FTP,又称为RunFTP。
  TeX目录只是其中一个文件夹,用来存放和TeX有关的软件、文档等资料;
  \item ftp://202.118.239.46/Incoming/TeX/  这个貌似是属于计算机系吧?(我不能确定,抱歉!) 目前由cliff@lilac等人维护,
建议此FTP上的TeX目录只是在241 出现问题down掉的时候采用,不过此FTP上的资料和241上的没有关系,二者是独立的。
\end{hitlist}

非常感谢上面两个FTP的管理员给TeX版提供软件和资料存放空间,并且给予维护管理。

%%% Local Variables: 
%%% mode: latex
%%% TeX-master: "../main"
%%% End: 
