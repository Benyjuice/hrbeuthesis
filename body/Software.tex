\defaultfont
\chapter{接收机系统设计}

\section{MSP432驱动软件设计}
	MSP432驱动程序包括ADC数据采集驱动和UART数据传输驱动两部分。
在设计驱动程序时参考了中断-回调函数模型,即中断服务函数处理中断请求然后请求调用
回调函数。回调函数不是由中断服务函数直接调用,而是由中断服务函数连接到回调函数链表中由
主函数根据优先级决定执行哪个回调函数。

	下面结合这一模型及具体驱动程序做详细说明。

\subsection{ADC数据采集驱动}
	系统采用带通采样以降低数据处理负荷,采样率$4KHz$。MSP432的Timer\_A0为ADC14提供采样率
的采样保持时钟。为保证系统实时性,采集数据设计为PING-PONG存储。MSP432的$\mu$DMA模块提供这一功能。
PING-PONG大小为$2\times1024$存储单元。每采集完成一次PING-PONG即触发一次中断,在中断服务函数中判断
是PING还是PONG,然后请求调用回调函数。

	回调函数的形式为:void msp432\_adc14\_callback(uint16\_t *data, uint32\_t len)。
参数data为当前完成采集的存储区的首地址,参数len为存储区大小。图\ref{Figure:Software:ADCModule}为ADC数据采集驱动的流程图。
	% 流程图定义基本形状
	\tikzstyle{startstop} = [rectangle, rounded corners, minimum width=3cm, minimum height=1cm,text centered, draw=black]
	\tikzstyle{io} = [trapezium, trapezium left angle=70, trapezium right angle=110, minimum width=3cm, minimum height=1cm, text centered, draw=black]
	\tikzstyle{process} = [rectangle, minimum width=3cm, minimum height=1cm, text centered, draw=black]
	\tikzstyle{decision} = [diamond, minimum width=3cm, minimum height=1cm, text centered, draw=black]
	\tikzstyle{arrow} = [thick,->,>=stealth]
	\begin{figure}[htbp]
		\centering
		\begin{tikzpicture}[node distance=2.0cm]
		%定义流程图具体形状
		\node (start) [startstop] {中断触发};
		\node (dec1) [decision, below of=start, yshift=-1.5cm] {PING中断?};
		\node (dec2) [decision, below of=dec1, yshift=-2.5cm] {PONG中断?};
		\node (pro1) [process, right of=dec2, xshift=2.5cm] {请求回调};
		\node (return) [startstop, below of=dec2,yshift=-1.5cm] {返回};
		\path node(fake) [right of=pro1]{};
		
		
		%连接具体形状
		\draw [arrow](start) -- (dec1);
		\draw [arrow](dec1) -| node[anchor=south] {是} (pro1);
		\draw [arrow](dec1) -- node[anchor=east] {否} (dec2);
		
		\draw [arrow](dec2) -- node[anchor=north] {是} (pro1);
		\draw [arrow](dec2) -- node[anchor=east] {否} (return);
		
		\draw [arrow](pro1) |-  (return);
		\end{tikzpicture}
		\caption{ADC数据采集驱动流程图}
		\label{Figure:Software:ADCModule}
	\end{figure}

\subsection{UART数据传输驱动}
	UART驱动设计为1152008N1(波特率115200bps,8位数据位,无校验位,1位停止位)传输格式。UART的接受
采用中断-回调函数模型,回调函数形式为:void msp432\_uart\_callback(uint16\_t *data, uint32\_t len)。

	UART的发送利用$\mu$DMA以实现非阻塞调用。提供给的外部编程接口为函数int msp432\_uart\_send(uint8\_t *data, uint32\_t len)
,函数将数据存入缓存区域并启动uDMA通道。函数返回实际写入缓存区域的数据长度,如果写入失败则返回-1。 

\section{MSP432动态内存管理}


	单片机在做信号处理算法时常常遇到需要使用较大数组的问题。例如在做32bit浮点采样数据为900点,本地数据为108点的快速相关算法时
需要分配至少两个4k大小的存储区域。在编写函数时有三种处理方式:

\begin{publist}
	\item 使用全局变量静态分配内存,存储区位于静态数据区。这样做的好处是分配内存在程序便以阶段就已经做好,但是内存永久占用不能释放,而且对于内存较小的
	单片机来说,分配较多的大块内存将带来大量内存的消耗;
	\item 使用局部变量,存储区位于栈内。这样做可以函数调用完之后内存可以释放,可以缓解静态分配内存的问题。但是如果遇到函数多次嵌套调用可能会导致栈益处;
	\item 使用malloc动态分配内存,存储区位于堆内。这样做可以解决静态分配和局部变量动态分配的所有问题,但是c标准库中的malloc函数算法复杂,分配与释放
	很耗时,而且需要链接较大的标准库的代码使程序体积增大。
\end{publist}

	信号处理算法虽然需要较多的大块内存,但是所有的内存大小几乎都一致即对某个固定大小的内存需求较多。针对这一前提,并结合动态内存分配的内存池算法重新实现了
快速的动态内存管理。


\subsection{内存池介绍}
	内存池(Memory Pool),又被称为固定大小区块规则(fixed-size-block allocation),允许程序设计者以类似C语言的malloc或者C++的new运算符进行动态内存的申请。
对于其他动态内存分配算法来说,因为会动态记忆区块大小导致的碎片问题,致使在实时系统上表现不佳,甚至根本无法使用。
内存池技术提供了一种更有效率的解决方案:预先规划一定数量的内存区块,使程序可以在执行期分配(allocate)、使用(access)和释放(free)内存区块。
	
	图\ref{Figure:Software:memalloc}为以内存池实例,内存池包含4个1k大小、2个2k大小和1个4k大小的内存块。图\ref{Figure:Software:memalloc}还展示了多次分配内存的过程:
	
\begin{publist}
	\item 请求一个大小为4k的内存快,此时内存池匹配到第一个大小为2k的内存区域;
	\item 请求4个大小都为1k的内存快,此时内存池分别匹配到4个大小为1k的内存区域,至此大小为1k的内存块用尽;
	\item 再次请求一个大小为1k的内存块,此时内存池将在2k大小的池内匹配内存区域
	\item 如果再请求1k或者2看的内存块,内存池将在4k大小的池内匹配内存区域。
\end{publist}

	\begin{figure}[htbp]
		\centering
		\def \nod{~~~~}
\def \zihao{\wuhao}
	\tikzstyle{neuron}=[circle,fill=black!25,minimum size=0pt,inner sep=0pt]
	\tikzstyle{num}=[circle,fill=red!25,minimum size=0pt,inner sep=0pt]

\begin{tikzpicture}[>=stealth',node distance=1cm]
	\draw 
		(0,0) rectangle ++(1,1)
		(1,0) rectangle ++(1,1)
		(2,0) rectangle ++(1,1)
		(3,0) rectangle ++(1,1)
		(4,0) rectangle ++(2,1)
		(6,0) rectangle ++(2,1)
		(8,0) rectangle ++(4,1)
		;
	\draw 
		(0.5,0.5) node {1k}
		++(1,0) node {1k}
		++(1,0) node {1k}
		++(1,0) node {1k}
		++(1.5,0) node {2k}
		++(2,0) node {2k}
		++(3,0) node {4k}
	;

	\draw[yshift=-1.5cm] 
		(0,0) rectangle ++(1,1)
		(1,0) rectangle ++(1,1)
		(2,0) rectangle ++(1,1)
		(3,0) rectangle ++(1,1)
		(4,0) rectangle ++(2,1)
		(6,0) rectangle ++(2,1)
		(8,0) rectangle ++(4,1)
	;
	\draw[yshift=-1.5cm,fill=gray!20] (4,0) rectangle ++(2,1);
	
	\draw[yshift=-3cm] 
		(0,0) rectangle ++(1,1)
		(1,0) rectangle ++(1,1)
		(2,0) rectangle ++(1,1)
		(3,0) rectangle ++(1,1)
		(4,0) rectangle ++(2,1)
		(6,0) rectangle ++(2,1)
		(8,0) rectangle ++(4,1)
	;
	\draw[yshift=-3cm,fill=gray!20]
		(0,0) rectangle ++(1,1)
		(1,0) rectangle ++(1,1)
		(2,0) rectangle ++(1,1)
		(3,0) rectangle ++(1,1)
		(4,0) rectangle ++(2,1)
	;
	
	\path[yshift=-4.5cm,fill=gray!20] (6,0) rectangle ++(1,1);
	\draw[yshift=-4.5cm] 
		(0,0) rectangle ++(1,1)
		(1,0) rectangle ++(1,1)
		(2,0) rectangle ++(1,1)
		(3,0) rectangle ++(1,1)
		(4,0) rectangle ++(2,1)
		(6,0) rectangle ++(2,1)
		(8,0) rectangle ++(4,1)
		;
	\draw[yshift=-4.5cm,fill=gray!20]
		(0,0) rectangle ++(1,1)
		(1,0) rectangle ++(1,1)
		(2,0) rectangle ++(1,1)
		(3,0) rectangle ++(1,1)
		(4,0) rectangle ++(2,1)
		;
	
	\draw node at (-1.cm,-1.cm) {\wuhao 分配2k};
	\draw[yshift=-1.5cm] node at (-1.4cm,-1.cm) {\wuhao 分配4个1k};
	\draw[yshift=-3cm] node at (-1.cm,-1.cm) {\wuhao 分配1k};
		
\end{tikzpicture}
		\caption{内存池内存分配实例}
		\label{Figure:Software:memalloc}
	\end{figure}

\subsection{内存池的优缺点}
	内存池允许在程序执行时以常数时间分配内存块,并且不会产生内存碎片。一次释放内存中大量空闲内存只需要一个操作,无需像malloc那样依次个别释放。
内存池不必将每次分配的内存的详细信息记录下来(例如内存大小,因为内存池本身就包含了大小的信息)。内存池在使用时也必须按照程序需求来做调整才能保证
时间与空间的效率,这也是内存池的显著缺点。

\subsection{内存池在本文中的应用}
	本文所涉及到的算法及所需要的内存大小如表\ref{Tab:Software:Tab1}所示。分析得出内存池需要最多1个32~Bytes,2个128Bytes,6个2048~Bytes和1个8192~Bytes的内存空间。
进一步分析,快速相关以及解调解扩相互间都是独立的,即前一算法执行完毕后一算法才会执行。因此,2048~Bytes的空间只需要2个。为给其他局部变量预留空间,最终的内存池分布为:
16~Bytes$\times$8、32~Bytes$\times$4、128~Bytes$\times$2、256~Bytes$\times$1、2048~Bytes$\times$2、4096~Bytes$\times$1、8192~Bytes$\times$1。


	\begin{table}[htbp]
		\centering 
		\caption{算法内存需求}
		\label{Tab:Software:Tab1}
		\vspace{0.5ex}
		\wuhao
		\begin{tabu} to \textwidth {X[1,c]|X[1,c]|X[1,c]|X[1,c]}
			\specialrule{1.5pt}{0pt}{0pt}
			% after \\: \hline or \cline{col1-col2} \cline{col3-col4} ...
			算法 & 		数据量 & 数据类型 & 占用空间(Byte) \\
			\hline
			快速相关& 	1024 $\times$ 2 & int16 & 2048 $\times$ 2 \\
			\hline
			\multirow{2}*{解调} & 2048$\times$2 & float32 & 8192$\times$2 \\
							& 512$\times$2  & float32 & 2048 $\times$2\\
			\hline
			\multirow{3}*{解扩与二阶差分检测} & 512 $\times$2 & float32 & 2048 $\times$ 2\\
							& 32 $\times$ 2 & float32 & 128 $\times$2 \\
							& 32 $\times$ 1 & int8 & 32 $\times$1 \\
			\specialrule{1.5pt}{0pt}{0pt}
		\end{tabu}
	\end{table}
	
\section{算法实现程序设计}
该部分程序主要实现DD-SS算法及数据流程控制,由同步、缓存、解扩三部分组成,它们之间的关系如图\ref{fig:software:ddss}所示。由ADC采集到的数据首先会进行同步搜索,搜索到同步信号后同步信号后的数据解调降采样后缓存。等到缓存满一帧后开始解扩及二阶差分检测,然后通过UART数据传输驱动将数据发出。
\begin{figure}[htbp]
	\centering
	\def \nod{~~~~}
\def \zihao{\wuhao}
	\tikzstyle{neuron}=[circle,fill=black!25,minimum size=0pt,inner sep=0pt]
	\tikzstyle{num}=[circle,fill=red!25,minimum size=0pt,inner sep=0pt]
	\tikzstyle{vecArrow} = [thick, decoration={markings,mark=at position
		1 with {\arrow[semithick]{open triangle 60}}},
	double distance=1.4pt, shorten >= 5.5pt,
	preaction = {decorate},
	postaction = {draw,line width=1.4pt, white,shorten >= 4.5pt}]
	\tikzstyle{adct} = [rectangle,fill=gray!25,minimum width=5cm, minimum height=1cm,text centered, draw=black]
	\tikzstyle{block} = [rectangle,fill=gray!75!red,minimum width=2cm, minimum height=1cm,text centered, draw=black]
\begin{tikzpicture}[>=stealth',node distance=3cm]
	\draw 
		node[adct] (adc) {ADC数据采集驱动}
		node[adct,xshift=6cm] (uart) {UART数据传输驱动}
		node[block,above of=adc,xshift=-1.5cm] (sync) {同步}
		node[block,right of=sync,xshift=1cm] (cache) {缓存}
		node[block,above of=uart] (pro) {解扩}
	;
	\node[inner sep=0,minimum size=0](hid) at (-1.5cm,0.5cm){};
	\node[inner sep=0,minimum size=0,right of=hid,xshift=-2.98cm,yshift=1cm](k) {};
	\draw [vecArrow] (hid) -- (sync);
	\draw [vecArrow] (k) -| (cache);
	\draw [vecArrow] (cache) -- (pro);
	\draw [vecArrow] (pro) -- (uart);
	\draw [vecArrow,dashed] (sync) -- (cache);
	
	\draw[dashed,very thick] (-4cm,1.0cm) --++(14cm,0);
	\node at (9.5cm,1.4cm) {应用层};
	\node at (9.5cm,0.6cm) {驱动层};
	
\end{tikzpicture}
	\caption{DD-SS程序组成框图}
	\label{fig:software:ddss}
\end{figure}

\subsection{同步搜索}
同步搜索采用双曲线调频信号(hyperbolic frequency-modulation,HFM)做相关进行。HFM是一种多普勒不变信号,其时间函数为\citeup{田坦2009声呐技术}
\begin{equation}
s(t)=Ae^{\left[-j\left(2\pi\frac{f_0^2}{m}\right)\ln\left(1-\frac{m}{f_0} t\right)\right]}
\end{equation}
频谱函数为\citeup{田坦2009声呐技术}
\begin{equation}
|S(f)|=A\frac{f_0}{\sqrt{|m|}\cdot f}
\end{equation}
图\ref{fig:software:hfm}展示了双曲线调频信号的幅度谱。
\begin{figure}[htbp]
	\centering
	\def \nod{~~~~}
\def \zihao{\wuhao}
\begin{tikzpicture}[>=stealth',node distance=3cm]
	\draw [->] (0,0) -- (6cm,0) node[below] {$f$};
	\draw [->] (0,0) -- (0,6cm) node[right] {$|S(f)|$};
	
	\draw[thick] plot[domain=1:5,samples=300,id=hfm] function{4.5/x};
	\draw[dashed] (1,0) -- (1,4.5)--(0,4.5);
	\draw[dashed] (3,0) -- (3,1.5)--(0,1.5);
	\draw[dashed] (5,0) -- (5,4.5/5)--(0,4.5/5);
	
	\node[left] at (0,4.5) {$\frac{Af_0}{\sqrt{|m|}f_L}$};
	\node[above left] at (0,1.5) {$\frac{A}{\sqrt{|m|}}$};
	\node[left] at (0,4.5/5) {$\frac{Af_0}{\sqrt{|m|}f_H}$};
	
	\node[below] at (1,0) {$f_L$};
	\node[below] at (3,0) {$f_0$};
	\node[below] at (5,0) {$f_H$};
	
	
	
\end{tikzpicture}
	\caption{双曲线调频信号幅度谱}
	\label{fig:software:hfm}
\end{figure}

由于HFM信号具有多普勒不变性\citeup{田坦2009声呐技术},即具有多普勒的接受信号$r(t)$与发射信号$s(t)$满足
\begin{equation}
r(t)=s(Dt)=s(t-t_0)
\end{equation}
式中,$D=1/\alpha\approx1+\delta$。可见,HFM将多普勒转化成为一个延时$t_0$,且
\begin{equation}
t_0=\frac{1-D}{(m/f_0)D}
\end{equation}

在MSP432的CMSIS-DSP库的直接相关算法计算速度慢,文献\cite{press2007numerical}介绍一种采用FFT的快速相关算法,利用快速相关算法可以节省大量运算时间。根据式(\ref{eq:ddcode:recv})可将接受信号写为
\begin{equation}
r(t)=\sum_{p=1}^{N_p}A_px(t-\tau_p)+n(t)
\end{equation}
式中$n(t)$为高斯白噪声,$x(t)$为HFM同步信号。经过相关后\citeup{farrokhi2008performance}
\begin{equation}
R_{rx}(\tau)=\sum_{p=1}^{N_p}A_pR_{xx}(\tau-\tau_p)+u(\tau)
\end{equation}
式中$u(\tau)$是噪声分量。$R_{rx}(\tau)$在$\tau_p(p=1,2,\cdots,N_p)$处有峰值,峰值$|A_p1R_{xx}(0)|$只与增益$A_p$有关。因此,选择峰值最大的第$q$个峰将得到最大的处理后信噪比。

\subsection{通带解调与缓存}
通带解调过程如图\ref{fig:software:pass}所示。I路与Q路分别乘以$cos(\omega n)$和$sin(\omega n)$后经过降采样滤波器得到时间宽度为$T_c$(扩频码宽度)的序列,然后缓存到缓存区。降采样滤波器将滤波和降采样结合,能够避免不必要计算的数据点大大节约了计算时间。缓存时,按照I路与路分开存储。
\begin{figure}[htbp]
	\centering
	
\begin{tikzpicture}[>=stealth']
		\tikzset{%
			block/.style    = {draw, thick, rectangle, minimum height = 2em,
				minimum width = 3em},
			sum/.style      = {draw, circle, inner sep=0pt, node distance = 2cm}, % Adder
			input/.style    = {coordinate}, % Input
			output/.style   = {coordinate} % Output
		}
		
		\node[input] (in) at(0,0) {};
		\node[sum] (t1) at (2,1) {\Large $\times$};
		\node[sum] (t2) at (2,-1) {\Large $\times$};
		\node[block,right of=t1,node distance = 2cm] (m1) {$\downarrow M$~FIR};
		\node[block,right of=t2,node distance = 2cm] (m2) {$\downarrow M$~FIR};
		\node[input,inner sep=0,minimum size=0] (f1) at(1,0) {};
		\draw (in) -- (f1);
		\draw[->] (f1) |- (t1);
		\draw[->] (f1) |- (t2);
		\draw[->] (t2) -- (m2);
		\draw[->] (t1) -- (m1);
		\draw node[above of=t1] (src1) {$\cos(\omega n)$};
		\draw[->] (src1)--(t1);
		
		\draw node[below of=t2] (src2) {$\sin(\omega n)$};
		\draw[->] (src2)--(t2);
		\draw node[above of=in,below] {$r(t)$};
		
		\node[inner sep=0] (save) at (6,0) {$\circledS$};
		\node[above of=save,below right] {\wuhao 复数形式缓存};
		
		\draw[->] (m1) -| (save);
		\draw[->] (m2) -| (save);
\end{tikzpicture}
	\caption{基带解调}
	\label{fig:software:pass}
\end{figure}

当缓存区满一帧时,调用解扩程序进行剩余的数据处理过程。

\subsection{解扩与解二阶差分}

	
	
	
	